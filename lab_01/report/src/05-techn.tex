\section{Технологическая часть}
В данном разделе приведены требования к программному обеспечению, средства реализации и листинги кода.

\subsection{Требования к ПО}

К программе предъявляется ряд требований:
\begin{itemize}
	\item на вход подаются две строки на русском или английском языке в любом регистре;
	\item на выходе -- искомое расстояние для трёх методов и их время выполнения.
\end{itemize}

\subsection{Средства реализации}

В качестве языка программирования для реализации лабораторной работы был выбран С++.

Данный выбор обусловлен поддержкой языком парадигмы объектно -- ориентированного программирования и наличием библиотек для точного замера времени.

Время выполнения реализации алгоритмов было замерено с помощью библиотеки \textit{chrono}.

\subsection{Сведения о модулях программы}
Программа состоит из трех программных модулей:
\begin{enumerate}[label={\arabic*)}]
	\item main.cpp -- главный модуль программы, содержащий функцию main и замер времени;
	\item algorithm.cpp, algorithm.h -- модуль с реализацией алгоритмов;
	\item menu.cpp, menu.h -- модуль с выбором режима работы программы;
	\item tests.cpp, tests.h -- модуль с тестированием программы.
\end{enumerate}

\newpage
\subsection{Реализация алгоритмов}

В листингe \ref{lst:algo1} приведена реализация итеративного алгоритма нахождения расстояния Левенштейна.


\begin{lstinputlisting}[
	caption={Итеративный алгоритм нахождения расстояния Левенштейна},
	label={lst:algo1}
	]{listings/lev.cpp}
\end{lstinputlisting}

\newpage
В листингe \ref{lst:algo2} приведена реализация итеративного алгоритма нахождения расстояния Дамерау-Левенштейна.

\begin{lstinputlisting}[
	caption={Итеративный алгоритм нахождения расстояния Дамерау-Левенштейна},
	label={lst:algo2}
	]{listings/dlev.cpp}
\end{lstinputlisting}

\newpage
В листингe \ref{lst:algo3} приведена реализация рекурсивного алгоритма нахождения расстояния Дамерау-Левенштейна.

\begin{lstinputlisting}[
	caption={Рекурсивный алгоритм нахождения расстояния Дамерау-Левенштейна},
	label={lst:algo3}
	]{listings/dlev_req.cpp}
\end{lstinputlisting}

\newpage
В листингe \ref{lst:algo4} приведена реализация рекурсивного алгоритма нахождения расстояния Дамерау-Левенштейна с кэшем.

\begin{lstinputlisting}[
	caption={Рекурсивный алгоритм нахождения расстояния Дамерау-Левенштейна с кэшем},
	label={lst:algo4}
	]{listings/dlev_req_cache.cpp}
\end{lstinputlisting}