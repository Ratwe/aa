\section*{ВВЕДЕНИЕ}
\addcontentsline{toc}{section}{ВВЕДЕНИЕ}

Чем более сложные задачи мы собираемся решать, тем большие требования предъявляются к временным характеристикам программ. 
Для сравнения: моделирование образования белка: потребует 1025 операций, что займет на одноядерном ПК тысячи
веков. 
А не менее актуальная задача прогноза погоды в масштабах всей планеты для получения прогноза на 10 дней потребуется 1016 операций с плавающей точкой, что составит примерно 10 дней. 
Параллельные алгоритмы в некоторых случаях способны уменьшить остроту ситуации \cite{крючкова2020программирование}. 

Цель данной лабораторной работы состоит в изучении основных принципов параллельных вычислений на основе нативных потоков и их применение для решения конкретной задачи.

Вариант 6: поиск подстроки в строке полным перебором (поиск одной подстроки в файле от 100 Мбайт) с заполнением одного несортированного результирующего файла с индексами строк и символов, где найдены вхождения.

Для достижения поставленной цели необходимо выполнить следующие задачи:
\begin{itemize}
	\item описать схему последовательного алгоритма поиска подстроки в строке методом полного перебора;
	\item разработать многопоточную версию данного алгоритма;
	\item описать схему алгоритма работы главного потока, который создаёт и запускает вспомогательные потоки;
	\item описать схему алгоритма вспомогательного потока;
	\item обосновать необходимость использования мьютексов и/или семафоров как примитивов синхронизации.
\end{itemize}