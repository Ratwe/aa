\section*{ЗАКЛЮЧЕНИЕ}
\addcontentsline{toc}{section}{ЗАКЛЮЧЕНИЕ}

Цель работы достигнута: описаны последовательные и параллельные конвейерные вычисления нахождения подстроки в строке стандартным алгоритмом и алгоритмом Кнута~---~Морриса~---~Пратта (КМП), проведено сравнение времени работы.

В ходе выполнения лабораторной работы были решены все задачи:
\begin{itemize}
	\item описана структура конвейерной обработки данных;
	\item описаны и реализованы используемые алгоритмы;
	\item определены средства программной реализации;
	\item реализована программа, выполняющая заданную конвейерную обработку;
	\item проведено сравнение реализаций алгоритмов по затраченному времени;
	\item подготовлен отчет по лабораторной работе.
\end{itemize}

В результате исследования выяснилось, что для задачи поиска подстроки в строке с помощью конвейерной обработки следует использовать последовательную реализацию, поскольку она быстрее параллельной при кол-ве заявок до 1000. 
В среднем выигрышь составляет 40\%.