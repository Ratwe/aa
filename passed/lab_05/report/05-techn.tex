\section{Технологическая часть}
В данном разделе приведены требования к программному обеспечению, средства реализации и листинги кода.

\subsection{Средства реализации}
В качестве языка программирования для реализации лабораторной работы был выбран C++.
Данный выбор обусловлен наличием функционала для работы с потоками thread и библиотекой chrono для замера времени выполнения алгоритмов.

\subsection{Сведения о модулях программы}
Программа состоит из следующих программных модулей:
\begin{itemize}
	\item main.cpp --- файл, содержащий меню и точку входа в программу;
	\item time.cpp --- модуль замера времени выполнения алгоритмов;
	\item prefix.cpp --- модуль расчёта массива префиксов;
	\item KMP.cpp --- модуль реализации алгоритма КМП;
	\item standart.cpp --- модуль реализации стандартного алгоритма;
	\item generate.cpp --- модуль генерации данных;
	\item conveyor.cpp --- модуль реализации конвейерной обработки;
	\item соответствующие им файлы заголовков;
	\item plot.py --- модуль создания графического представления значений.
\end{itemize}

\newpage
\subsection{Реализация алгоритмов}

В листингe \ref{lst:standart} приведена реализация стандартного алгоритма поиска подстроки в строке.

\begin{lstinputlisting}[
	label={lst:standart},
	caption={Стандартный алгоритм},
	firstline=11,
	lastline=26
	]{../src/lab_05/standart.cpp}
\end{lstinputlisting}

В листингe \ref{lst:prefix} приведена реализация алгоритма расчета префиксного массива, необходимого для алгоритма КМП.

\begin{lstinputlisting}[
	label={lst:prefix},
	caption={Расчет префиксного массива},
	firstline=8,
	lastline=23
	]{../src/lab_05/prefix.cpp}
\end{lstinputlisting}

В листингe \ref{lst:KMP} приведена реализация алгоритма КМП поиска подстроки в строке.

\begin{lstinputlisting}[
	label={lst:KMP},
	caption={Алгоритм КМП},
	firstline=11,
	lastline=33
	]{../src/lab_05/KMP.cpp}
\end{lstinputlisting}
\newpage

В листингe \ref{lst:linear} приведена последовательная реализация конвейера.

\begin{lstinputlisting}[
	label={lst:linear},
	caption={Последовательный конвейер},
	firstline=14,
	lastline=26
	]{../src/lab_05/conveyor.cpp}
\end{lstinputlisting}

В листингах \ref{lst:parallel} и \ref{lst:parallel2} приведена параллельная реализация конвейера.

\begin{lstinputlisting}[
	label={lst:parallel},
	caption={Параллельный конвейер},
	firstline=28,
	lastline=45
	]{../src/lab_05/conveyor.cpp}
\end{lstinputlisting}

\newpage

\begin{lstinputlisting}[
	label={lst:parallel2},
	caption={Продолжение листинга \ref{lst:parallel}},
	firstline=47,
	lastline=66
	]{../src/lab_05/conveyor.cpp}
\end{lstinputlisting}

В листингe \ref{lst:sub} приведена реализация алгоритма генерации искомой подстроки.

\begin{lstinputlisting}[
	label={lst:sub},
	caption={Генерация подстроки},
	firstline=7,
	lastline=18
	]{../src/lab_05/generate.cpp}
\end{lstinputlisting}

\newpage
В листингe \ref{lst:str} приведена реализация алгоритма генерации строки.

\begin{lstinputlisting}[
	label={lst:str},
	caption={Генерация строки},
	firstline=20,
	lastline=40
	]{../src/lab_05/generate.cpp}
\end{lstinputlisting}