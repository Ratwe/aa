\section*{ЗАКЛЮЧЕНИЕ}
\addcontentsline{toc}{section}{ЗАКЛЮЧЕНИЕ}

Цель работы достигнута: составлен отчёт и программа для умножения матриц с использованием:
\begin{itemize}
	\item классического алгоритма;
	\item алгоритма Винограда;
	\item алгоритма Штрассена;
	\item их же с оптимизациями.
\end{itemize}

В ходе выполнения лабораторной работы были решены все задачи:
\begin{itemize}
	\item реализованы требуемые алгоритмы;
	\item проведено сравнение требуемого времени выполнения реализуемых алгоритмов в тактах процессора и занимаемой памяти;
	\item подготовлен отчёт по лабораторной работе.
\end{itemize}

Замеры показали, что алгоритм Винограда и его оптимизированная версия являются самыми эффективными \textit{по времени} начиная с длины квадратных матриц, равной четырём.
Классический алгоритм эффективнее других \textit{по времени} только для случаев перемножения матриц длинной 1 и 3.
Алгоритм Штрассена является самым эффективным \textit{по памяти}, начиная с длины матриц, равной двум.

Рассчитанная трудоёмкость классического алгоритма перемножения матриц меньше, чем трудоёмкость алгоритма Винограда.
Но, не смотря на это, начиная с длины квадратных матриц, равной четырём, время выполнения реализации классического алгоритма становится больше, чем время выполнения реализации алгоритма Винограда.