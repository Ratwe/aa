\section{Технологическая часть}
В данном разделе приведены требования к программному обеспечению, средства реализации и листинги кода.

\subsection{Требования к ПО}
На вход подаются две матрицы.
На выходе требуется получить результирующую матрицу -- произведение двух введённых.

\subsection{Средства реализации}

В качестве языка программирования для реализации лабораторной работы был выбран С++.
Данный выбор обусловлен наличием инструкции \cite{microsoft_rdtsc} для замера количества тактов и поддержкой ООП.

\subsection{Сведения о модулях программы}
Программа состоит из трех программных модулей:
\begin{enumerate}[label={\arabic*)}]
	\item main.cpp -- главный модуль программы, содержащий меню;
	\item algo.cpp, algo.h -- модуль с реализацией алгоритмов;
	\item io\_matrix.cpp, io\_matrix.h -- модуль ввода/вывода матриц;
	\item time.cpp, time.h -- модуль замера времени работы алгоритмов.
\end{enumerate}

\newpage



\subsection{Реализация алгоритмов}

В листингe \ref{lst:classic} приведена реализация классического алгоритма умножения двух матриц.

\begin{lstinputlisting}[
	label={lst:classic},
	caption={Классический алгоритм},
	]{../src/lab_02/classic.cpp}
\end{lstinputlisting}

В листингe \ref{lst:winograd} приведена реализация алгоритма Винограда умножения двух матриц.

Реализация оптимизированного алгоритма Винограда идентична реализации \textit{обычного} алгоритма Винограда, за исключением замены операции умножения на 2 на побитовый сдвиг и предвычисления \code{half\_m = m / 2}.

В этом методе переменная \code{optimized} отвечает за то, какую реализацию использовать.
\begin{equation}
	\label{for:opt}
	mode =
	\begin{cases}
		\text{алгоритм Винограда}, & \text{optimized = 0}\\
		\text{оптимизированный алгоритм Винограда}, & \text{иначе.}
	\end{cases}
\end{equation}

\newpage

\begin{lstinputlisting}[
	label={lst:winograd},
	caption={Алгоритм Винограда},
	]{../src/lab_02/winograd.cpp}
\end{lstinputlisting}

\newpage 

\begin{lstinputlisting}[
	label={lst:winograd2},
	caption={продолжение листинга \ref{lst:winograd}},
	]{../src/lab_02/winograd2.cpp}
\end{lstinputlisting}

\newpage

В листингe \ref{lst:strassen} приведена реализация алгоритма Штрассена умножения двух матриц.

\begin{lstinputlisting}[
	label={lst:strassen},
	caption={Алгоритм Штрассена},
	]{../src/lab_02/strassen.cpp}
\end{lstinputlisting}

\newpage

\begin{lstinputlisting}[
	label={lst:strassen2},
	caption={продолжение листинга \ref{lst:strassen}},
	]{../src/lab_02/strassen2.cpp}
\end{lstinputlisting}