\section{Аналитическая часть}

<<<<<<< HEAD
Классический (стандартный) алгоритм перемножения матриц --- реализация математического определения умножения матриц.
=======
<<<<<<< HEAD
<<<<<<< HEAD
Классический (стандартный) алгоритм перемножения матриц --- реализация математического определения умножения матриц.
=======
Классический (стандартный) алгоритм перемножения матриц - реализация математического определения умножения матриц.
>>>>>>> 7ad89dc (lab_02 almost passed)
=======
Классический (стандартный) алгоритм перемножения матриц --- реализация математического определения умножения матриц.
>>>>>>> af69c2c (lab_02 passed)
>>>>>>> fb5d8bb (lab_02 passed)
Имеет ассимптотическую сложность \(O(n^{3})\).

Алгоритм Винограда имеет ассимптотическую сложность \(O(n^{2.3755})\), поэтому является одним из самых эффективных по времени алгоритмом умножения матриц \cite{Виноград}. 

Алгоритм Штрассена имеет ассимптотическую сложность \(O(n^{2.78})\) \cite{Ultra-Fast}. 
Идея этого метода состоит в открытии того, что произведение С двух матриц А и В размером 2×2 можно вычислить с помощью только семи, а не восьми умножений, которые необходимы при использовании стандартного алгоритма \cite{Штрассен}.