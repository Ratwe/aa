\section{Аналитическая часть}

\subsection{Стандартный алгоритм}

Одним из самых очевидных и одновременно неэффективных алгоритмов является алгоритм последовательно (прямого) поиска. 
Суть его заключается в сравнении искомой подстроки с каждым словом в основной строке. 
Алгоритм не является эффективным, так как максимальное количество сравнений будет равно O((n-m+1)*m), где большинство из них на самом деле лишние. 
Для небольших строк поиск работает довольно быстро, но если в файлах с большим количеством информации последовательность символов будет искаться очень долго~\cite{вирт2010алгоритмы}.


\subsection{Алгоритм Кнута --- Морриса --- Пратта}
Метод, использующий предобработку искомой строки и создающий на ее основе префикс-функцию, содержится в алгоритме Кнута-Морриса-Пратта (КМП). 
Суть этой функции заключается в нахождении наибольшей подстроки, одновременно находящейся и в начале, и в конце подстроки (как префикс и как суффикс). 
Смысл префикс-функции заключается в том, что неверные варианты могут быть заранее отброшены, а в начале работы могут рассматриваться некоторые вспомогательные утверждения, где для произвольного слова рассматриваются все его начала, которые по совместительству являются его концами, и выбираются из них самое длинное. 
Метод КМП использует следующую идею: если префикс (он же суффикс) строки длиной i длиннее одного символа, то он одновременно и префикс подстроки длиной i-1. Время работы всей процедуры линейно и есть O(m), несмотря на то, что в ней присутствует вложенный цикл \cite{солдатова2018основные}.

