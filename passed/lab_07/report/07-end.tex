\section*{ЗАКЛЮЧЕНИЕ}
\addcontentsline{toc}{section}{ЗАКЛЮЧЕНИЕ}

Цель работы достигнута: проведено исследование лучших и худших случаев работы алгоритмов поиска подстроки в строке стандартным алгоритмом и Кнута~--~Морриса~--~Пратта.

В ходе выполнения лабораторной работы были решены все задачи:
\begin{itemize}
	\item описаны используемые алгоритмы поиска;
	\item выбраны средства программной реализации;
	\item реализованы данные алгоритмы поиска;
	\item проанализированы алгоритмы по количеству сравнений;
	\item подготовлен отчет по лабораторной работе.
\end{itemize}

В результате анализа замеров времени выполнения алгоритмов был сделан вывод, что, начиная с размера текста, равного 1000, алгоритм КМП работает быстрее стандартного алгоритма.
Для текста размера меньшего, чем 256 символов, рекомендуется использовать стандартный алгоритм.