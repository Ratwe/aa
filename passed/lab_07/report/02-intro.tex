\section*{ВВЕДЕНИЕ}
\addcontentsline{toc}{section}{ВВЕДЕНИЕ}

Задача поиска подстроки одна из достаточно распространённых в информатике. Строкой называют последовательность символов (в произвольном порядке) взятых из заданного алфавита. Так например, алфавитом могут быть цифры {0, 1} из которых можно составлять неограниченную по длине цепочку символов, например, 0110100110 или 0~\cite{max}. 

Практическое значение задачи о точных совпадениях трудно преувеличить. 
Эта задача возникает в широком спектре приложений: текстовых редакторах, информационно-поисковых системах, электронных библиотеках, каталогах и справочниках и т. д. 
Алгоритмы поиска подстроки в строке также применяются при поиске заданных образцов в молекулах ДНК~\cite{алексеенко2010информационная}.


Целью данной лабораторной работы является исследование лучших и худших случаев работы алгоритмов поиска подстроки в строке стандартным алгоритмом и Кнута~--~Морриса~--~Пратта.

Для достижения поставленной цели необходимо решить следующие задачи:
\begin{itemize}
	\item описать используемые алгоритмы поиска;
	\item выбрать средства программной реализации;
	\item реализовать данные алгоритмы поиска;
	\item проанализировать алгоритмы по количеству сравнений;
	\item подготовить отчет по лабораторной работе.
\end{itemize}
