\section{Технологическая часть}
В данном разделе приведены требования к программному обеспечению, средства реализации и листинги кода.

\subsection{Требования к ПО}
На вход подается строка и искомая в ней подстрока.
На выходе требуется получить индекс первого вхождения подстроки или отрицательное число, если такое не было найдено.

\subsection{Средства реализации}

В качестве языка программирования для реализации лабораторной работы был выбран Python.
Данный выбор обусловлен наличием библиотеки $time$~\cite{time} для замера времени выполнения алгоритмов.

\subsection{Сведения о модулях программы}
Программа состоит из следующих программных модулей:
\begin{enumerate}[label={\arabic*)}]
	\item standart.py --- модуль реализации стандартного алгоритма;
	\item prefix.py --- модуль реализации расчета префиксного массива;
	\item KMP.py --- модуль реализации алгоритма Кнута~--~Морриса~--~Пратта;
	\item time.py --- модуль замера времени выполнения алгоритмов;
	\item plot.py --- модуль создания графического представления значений;
	\item main.py --- модуль, содержащий меню.
\end{enumerate}

\newpage
\subsection{Реализация алгоритмов}

В листингe \ref{lst:standart} приведена реализация стандартного алгоритма поиска подстроки в строке.

\begin{lstinputlisting}[
	label={lst:standart},
	caption={Стандартный алгоритм},
	]{../src/lab_07/standart.py}
\end{lstinputlisting}

В листингe \ref{lst:prefix} приведена реализация расчета префиксного массива.

\begin{lstinputlisting}[
	label={lst:prefix},
	caption={Получение префиксного массива},
	]{../src/lab_07/prefix.py}
\end{lstinputlisting}

\newpage
В листингe \ref{lst:kmp} приведена реализация алгоритма Кнута~--~Морриса~--~Пратта.
\begin{lstinputlisting}[
	label={lst:kmp},
	firstline=4,
	caption={Алгоритм Кнута~--~Морриса~--~Пратта},
	]{../src/lab_07/kmp.py}
\end{lstinputlisting}

Алгоритм также возвращает второе значение кол-ва сравнений потому что это требуется для определения худшего случая положения искомой подстроки в тексте в исследователькой части.