\section*{ВВЕДЕНИЕ}
\addcontentsline{toc}{section}{ВВЕДЕНИЕ}

Целью данной лабораторной работы является изучение расстояния Левенштейна и Дамерау~---~Левенштейна.

Расстояние Левенштейна (редакционное расстояние) и его модификация - расстояние Дамерау~---~Левенштейна, представляют собой метрики, используемые для измерения различий между двумя последовательностями символов.
Они определяют минимальное количество односимвольных операций (вставка, удаление, замена и транспозиция), необходимых для преобразования одной последовательности символов в другую.
Эти метрики были разработаны советским математиком Владимиром Левенштейном в 1965 году и модифицированы впоследствии с учетом операции транспозиции символов, получив название расстояния Дамерау--Левенштейна.

Расстояние Левенштейна и его модификация имеют широкий спектр применений в различных областях, включая компьютерную лингвистику (автозамена, исправление ошибок), биоинформатику (анализ генома, белковых последовательностей).

Задачи лабораторной работы:
\begin{itemize}
    \item реализация алгоритмов с использованием динамического программирования;
    \item сравнение требуемого времени выполнения в тактах процессора и занимаемой памяти;
    \item подготовка отчёта по лабораторной работе.
\end{itemize}

