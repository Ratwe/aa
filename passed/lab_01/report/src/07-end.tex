\section*{ЗАКЛЮЧЕНИЕ}

Цель работы достигнута: изучены алгоритмы нахождения расстояния Левенштейна и Дамерау~---~Левенштейна.

В ходе выполнения лабораторной работы были решены все задачи:
\begin{itemize}
	\item изучены алгоритмы нахождения расстояния Левенштейна и Дамерау~---~Левенштейна;
	\item применены методы динамического программирования для реализации алгоритмов;
	\item на основе полученных в ходе экспериментов данных были сделаны выводы по поводу эффективности всех реализованных алгоритмов;
	\item был подготовлен отчет по лабораторной работе.
\end{itemize}

<<<<<<< HEAD
<<<<<<< HEAD:passed/lab_01/report/src/07-end.tex
Эксперименты показали, что наиболее затратный по времени рекурсивный алгоритм поиска расстояния Дамерау~---~Левенштейна без кеша, а наименее затратны итеративные алгоритмы. Менее затратными по памяти являются реализации итеративных алгоритмов. 
Самым затратным по памяти является реализация рекурсивного алгоритма поиска расстояния Дамерау~---~Левенштейна с кешированием. 
=======
Эксперименты показали, что наиболее затратный по времени рекурсивный алгоритм поиска расстояния Дамерау--Левенштейна без кеша, а наименее затратны итеративные алгоритмы. Менее затратными по памяти являются реализации итеративных алгоритмов. 
Самым затратным по памяти является реализация рекурсивного алгоритма поиска расстояния Дамерау--Левенштейна с кешированием. 
>>>>>>> 786b864 (lab_01 almost passed):lab_01/report/src/07-end.tex
=======
Эксперименты показали, что наиболее затратный по времени рекурсивный алгоритм поиска расстояния Дамерау~---~Левенштейна без кеша, а наименее затратны итеративные алгоритмы. Менее затратными по памяти являются реализации итеративных алгоритмов. 
Самым затратным по памяти является реализация рекурсивного алгоритма поиска расстояния Дамерау~---~Левенштейна с кешированием. 
>>>>>>> f0c1fd9 (lab_01 passed)
