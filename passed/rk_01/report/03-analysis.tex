\section{Аналитическая часть}

\subsection{Стандартный алгоритм}
Одним из самых очевидных и одновременно неэффективных алгоритмов является алгоритм последовательно (прямого) поиска. 
Суть его заключается в сравнении искомой подстроки с каждым словом в основной строке. 
Алгоритм не является эффективным, так как максимальное количество сравнений будет равно O((n-m+1)*m), где большинство из них на самом деле лишние. 
Для небольших строк поиск работает довольно быстро, но если в файлах с большим количеством информации последовательность символов будет искаться очень долго~\cite{вирт2010алгоритмы}.

Однако, на скорость работы алгоритма можно повлиять, если использовать идею распараллеливания потоков.
В данной работе реализован последовательный и параллельный подход.
При параллельном используется два потока: один обрабатывает первую половину файла, второй --- другую.