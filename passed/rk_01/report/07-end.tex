\section*{ЗАКЛЮЧЕНИЕ}
\addcontentsline{toc}{section}{ЗАКЛЮЧЕНИЕ}

Цель работы достигнута: создано ПО с последовательной однопоточной и параллельной двупоточной реализацией нахождения подстроки в строке стандартным алгоритмом.

В ходе выполнения лабораторной работы были решены все задачи:
\begin{itemize}
	\item описан стандартный алгоритм;
	\item определены средства программной реализации;
	\item реализован последовательный и параллельный вариант стандартного алгоритма;
	\item проведено сравнение обоих версий алгоритма;
	\item подготовлен отчет по лабораторной работе.
\end{itemize}

В результате исследования выяснилось, что для задачи поиска подстроки в строке с помощью стандартного алгоритма следует отдать предпочтение параллельной реализации, поскольку она быстрее последовательной при любом кол-ве заявок.
В среднем выигрышь составляет 33.89\% для кол-ва заявок до 100 штук.