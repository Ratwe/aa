\section{Технологическая часть}
В данном разделе приведены требования к программному обеспечению, средства реализации и листинги кода.

\subsection{Средства реализации}
В качестве языка программирования для реализации лабораторной работы был выбран C++.
Данный выбор обусловлен наличием функционала для работы с потоками thread и библиотекой chrono для замера времени выполнения алгоритмов.

\subsection{Сведения о модулях программы}
Программа состоит из следующих программных модулей:
\begin{itemize}
	\item main.cpp --- файл, содержащий меню и точку входа в программу;
	\item time.cpp --- модуль замера времени выполнения алгоритмов;
	\item standart.cpp --- модуль реализации стандартного алгоритма;
	\item generate.cpp --- модуль генерации данных;
	\item implementation.cpp --- модуль реализации двух видов алгоритма;
	\item соответствующие им файлы заголовков;
	\item plot.py --- модуль создания графического представления значений.
\end{itemize}

\newpage
\subsection{Реализация алгоритмов}

В листингe \ref{lst:standart} приведена реализация стандартного алгоритма поиска подстроки в строке.

\begin{lstinputlisting}[
	label={lst:standart},
	caption={Стандартный алгоритм},
	firstline=11,
	lastline=26
	]{../src/rk_01/standart.cpp}
\end{lstinputlisting}

В листингах \ref{lst:linear} приведена последовательная реализация программы.

\begin{lstinputlisting}[
	label={lst:linear},
	caption={Последовательная реализация},
	firstline=14,
	lastline=23
	]{../src/rk_01/implementation.cpp}
\end{lstinputlisting}

\newpage
В листингах \ref{lst:parallel} приведена параллельная реализация программы.

\begin{lstinputlisting}[
	label={lst:parallel},
	caption={Параллельная реализация},
	firstline=25
	]{../src/rk_01/implementation.cpp}
\end{lstinputlisting}

В листингe \ref{lst:sub} приведена реализация алгоритма генерации искомой подстроки.

\begin{lstinputlisting}[
	label={lst:sub},
	caption={Генерация подстроки},
	firstline=7,
	lastline=14
	]{../src/rk_01/generate.cpp}
\end{lstinputlisting}

\newpage
В листингe \ref{lst:str} приведена реализация алгоритма генерации строки.

\begin{lstinputlisting}[
	label={lst:str},
	caption={Генерация строки},
	firstline=16,
	lastline=36
	]{../src/rk_01/generate.cpp}
\end{lstinputlisting}