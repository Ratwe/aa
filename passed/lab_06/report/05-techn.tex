\section{Технологическая часть}
В данном разделе приведены требования к программному обеспечению, средства реализации и листинги кода.

\subsection{Требования к ПО}
На вход подаётся матрица стоимостей.
На выходе требуется получить минимальный путь --- последовательность вершин, которые необходимо посетить.
Т.к. по варианту необходимо найти Гамильтонов путь, первая и последняя вершина в последовательности всегда должна совпадать.

\subsection{Средства реализации}

В качестве языка программирования для реализации лабораторной работы был выбран Python.
Данный выбор обусловлен наличием библиотеки $time$~\cite{time} для замера времени выполнения алгоритмов.

\subsection{Сведения о модулях программы}
Программа состоит из следующих программных модулей:
\begin{enumerate}[label={\arabic*)}]
	\item matrix.py --- модуль с функциями обработки матриц;
	\item ant.py --- модуль с реализацией муравьиного алгоритма;
	\item bruteforce.py --- модуль с реализацией метода полного перебора;
	\item main.py --- главный модуль с меню и функцией параметризации;
	\item time.py --- модуль замера времени алгоритмов;
	\item plot.py --- модуль создания графического представления значений.
\end{enumerate}

\newpage
\subsection{Реализация алгоритмов}

В листингe \ref{lst:bruteforce} приведена реализация метода полного перебора.

\begin{lstinputlisting}[
	label={lst:bruteforce},
	caption={Полный перебор},
	]{../src/lab_06/bruteforce.py}
\end{lstinputlisting}

\newpage
В листингe \ref{lst:ant1} приведена реализация муравьиного алгоритма.

\begin{lstinputlisting}[
	label={lst:ant1},
	caption={Муравьиный алгоритм},
	firstline=1,
	lastline=37
	]{../src/lab_06/ant_nc.py}
\end{lstinputlisting}

\newpage
\begin{lstinputlisting}[
	label={lst:ant2},
	caption={продолжение листинга \ref{lst:ant1}},
	firstline=38,
	lastline=74
	]{../src/lab_06/ant_nc.py}
\end{lstinputlisting}

\newpage
\begin{lstinputlisting}[
	label={lst:ant3},
	caption={продолжение листинга \ref{lst:ant2}},
	firstline=75,
	lastline=89
	]{../src/lab_06/ant_nc.py}
\end{lstinputlisting}

\newpage
\begin{lstinputlisting}[
	label={lst:ant4},
	caption={продолжение листинга \ref{lst:ant3}},
	firstline=92,
	lastline=113
	]{../src/lab_06/ant_nc.py}
\end{lstinputlisting}

\newpage
