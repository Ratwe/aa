\section*{ЗАКЛЮЧЕНИЕ}
\addcontentsline{toc}{section}{ЗАКЛЮЧЕНИЕ}

Цель работы достигнута: проведен сравнительный анализ метода полного перебора и метода на базе муравьиного алгоритма в соответствии с вариантом.

В ходе выполнения лабораторной работы были решены все задачи:
\begin{itemize}[label*=---]
	\item описан схемой алгоритма и реализован метод полного перебора для решения задачи коммивояжера;
	\item описан схемой алгоритма и реализован метод решения задачи коммивояжёра на основе муравьиного алгоритма;
	\item указаны преимущества и недостатки реализованных методов;
	\item проведён сравнительный анализ двух рассмотренных методов решения задачи коммивояжера;
	\item подготовлен отчет по лабораторной работе.
\end{itemize}

В результате анализа замеров времени выполнения был сделан вывод, что, начиная с размера матрицы, равного пяти, алгоритм полного перебора работает значительно медленнее муравьиного алгоритма.

Также, для представленного класса данных лучшие результаты муравьиный алгоритм дает на значениях параметров, предоставленных в таблице~\ref{tab:res}, т.к. в результате параметризации алгоритм на всех графах класса безошибочно находил лучший путь.
\begin{table}[hbtp]
	\centering
	\caption{Результат выборки параметров}
	\label{tab:res}
	\begin{tabular}{|c|c|c|}
		\hline
		\textbf{$\alpha$} & \textbf{$\rho$} & {Итерации, шт.} \\
		\hline
		0.4 & 0.2 & 300 \\
		0.4 & 0.7 & 300 \\
		0.7 & 0.7 & 300 \\
		0.7 & 0.7 & 500 \\
		0.1 & 0.3 & 300 \\
		\hline
	\end{tabular}
\end{table}