\section*{ВВЕДЕНИЕ}
\addcontentsline{toc}{section}{ВВЕДЕНИЕ}

В 1831 г. в Германии вышла книга под названием «Кто такой коммивояжер и что он должен делать для процветания своего предприятия».
Одна из рекомендаций этой книги гласила: «Важно посетить как можно больше мест возможного сбыта, не посещая ни одно из них дважды».
По-видимому, это была первая формулировка задачи коммивояжера. 
Наиболее употребляемая формулировка ЗК состоит в следующем. 
Заданы список городов некоторого региона и таблица попарных расстояний между ними.
Требуется найти замкнутый (т.~е. начинающийся и заканчивающийся в одном и том же городе) маршрут коммивояжера, проходящий через все города, причем входящий в каждый город и выходящий из каждого города по одному разу и имеющий минимальную длину~\cite{меламед1989задача}.

Такая задача может быть решены при помощи полного перебора вариантов и эвристических алгоритмов.
Алгоритмы, основанные на использовании эвристического метода, не всегда приводят к оптимальным решениям.
Однако для их применения на практике достаточно, чтобы ошибка прогнозирования не превышала допустимого значения.

Целью данной лабораторной работы является сравнительный анализ метода полного перебора и метода на базе муравьиного алгоритма в соответствии с вариантом.

Вариант: ор-граф, с элитными муравьями, карта перемещения для воздухоплавателей (время с учётом направления --- по ветру или против ветра; горы нужно огибать; в привязке к карте). 
Гамильтонов цикл.

Задачи лабораторной работы:
\begin{itemize}[label*=---]
	\item описать схемой алгоритма и реализовать метод полного перебора для решения задачи коммивояжера;
	\item описать схемой алгоритма и реализовать метод решения задачи коммивояжёра на основе муравьиного алгоритма;
	\item указать преимущества и недостатки реализованных методов;
	\item провести сравнительный анализ двух рассмотренных методов решения задачи коммивояжера;
	\item подготовить отчет по лабораторной работе.
\end{itemize}