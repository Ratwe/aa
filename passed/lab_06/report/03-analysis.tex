\section{Аналитическая часть}

\subsection{Алгоритм полного перебора}

Для решения задачи коммивояжера алгоритм полного перебора предполагает рассмотрение всех возможных путей в графе и выбор наименьшего из них. 
Смысл перебора состоит в том, что перебираются все варианты объезда городов и выбирается оптимальный, что гарантирует точное решение задачи. 
Однако, при таком подходе количество возможных маршрутов факториально возрастает --- сложность алгоритма равна $n!$.


\subsection{Муравьиный алгоритм}
Муравьиный алгоритм --- метод решения задач коммивояжера на основании моделирования поведения колонии муравьев.

Каждый муравей определяет для себя маршрут, который необходимо пройти на основе феромона, который он получает во время прохождения. Каждый муравей оставляет феромон на своем пути, чтобы остальные муравьи по нему ориентировались. 
В результате при прохождении каждым муравьем различного маршрута наибольшее число феромона остается на оптимальном пути.
Распределение феромона по пространству передвижения муравьев является своего рода динамически изменяемой глобальной памятью муравейника. 
Любой муравей в фиксированный момент времени может воспринимать и изменять лишь одну локальную ячейку этой глобальной памяти \cite{штовба2003муравьиные}.

