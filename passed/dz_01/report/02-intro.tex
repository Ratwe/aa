\chapter{Введение}
\section{Задание}

Описать четырьмя графовыми моделями (граф управления, информационный граф, опрерационная история, информационная история) последовательный алгоритм либо фрагмент алгоритма, содержащий от 15 значащих строк кода и от двух циклов, один из которых является вложенным в другой.

\textbf{Вариант 6:} в качестве реализуемого алгоритма~--- поиск подстроки в файле.

\section{Графовые модели программы}

Программа представлена в виде графа: набор вершин и множество соединяющих их направленных дуг.

\begin{enumerate}
	\item \textbf{Вершины}: процедуры, циклы, линейный участки, операторы, итерации циклов, срабатывание операторов и т. д.
	\item \textbf{Дуги}: отражают связь (отношение между вершинами). 
\end{enumerate}

Выделяют 2 типа отношений:
\begin{enumerate}
	\item операционное отношение~--- по передаче управления;
	\item информационное отношение~--- по передаче данных.
\end{enumerate}

Граф управления~--- модель, в который \textbf{вершины}~---операторы, \textbf{дуги}~--- операционные отношения.

Информационный граф~--- модель, в которой \textbf{вершины}: операторы, \textbf{дуги}~--- информационные отношения.

Операционная история~--- модель, в которой \textbf{вершины}: срабатывание операторов, \textbf{дуги}~--- операционные отношения.

Информационная история~--- модель, в которой \textbf{вершины}: срабатывание операторов, \textbf{дуги}~--- информационные отношения.