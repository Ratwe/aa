\section{Технологическая часть}
В данном разделе приведены требования к программному обеспечению, средства реализации и листинги кода.

\subsection{Требования к ПО}
На вход подаётся массив.
На выходе требуется получить отсортированный массив.

\subsection{Средства реализации}

В качестве языка программирования для реализации лабораторной работы был выбран С++.
Данный выбор обусловлен наличием инструкции rdtsc \cite{microsoft_rdtsc} для замера количества тактов.

\subsection{Сведения о модулях программы}
Программа состоит из трех программных модулей:
\begin{enumerate}[label={\arabic*)}]
	\item main.cpp -- главный модуль программы, содержащий меню;
	\item algo.cpp, algo.h -- модуль с реализацией алгоритмов;
	\item time.cpp, time.h -- модуль замера времени работы алгоритмов.
\end{enumerate}

\subsection{Реализация алгоритмов}

В листингe \ref{lst:selection} приведена реализация алгоритма сортировки выбором.

\begin{lstinputlisting}[
	label={lst:selection},
	caption={Сортировка выбором},
	]{../src/lab_03/selection.cpp}
\end{lstinputlisting}

\newpage
В листингe \ref{lst:shaker} приведена реализация алгоритма сортировки перемешиванием.

\begin{lstinputlisting}[
	label={lst:shaker},
	caption={Сортировка перемешиванием},
	]{../src/lab_03/shaker.cpp}
\end{lstinputlisting}

\newpage
