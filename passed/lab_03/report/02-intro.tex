\section*{ВВЕДЕНИЕ}
\addcontentsline{toc}{section}{ВВЕДЕНИЕ}

Сортировка --- это последовательное расположение или разбиение на группы чего-либо в зависимости от выбранного критерия \cite{вашинко2018блинная}.

Любой алгоритм сортировки можно разбить на три основные части:
\begin{itemize}
	\item сравнение элементов для определения их упорядоченности;
	\item перестановка элементов;
	\item сортирующий алгоритм, который осуществляет сравнение и перестановку элементов до тех пор, пока все элементы не будут упорядочены.
\end{itemize}
	
Цель лабораторной работы –-- изучить и исследовать трудоемкость алгоритмов сортировки.

Для достижения поставленных целей ставятся задачи:

\begin{itemize}
	\item реализация требуемых алгоритмов сортировки: блинная, выбором, перемешиванием;
	\item сравнение требуемого времени выполнения реализуемых алгоритмов в тактах процессора и занимаемой памяти;
	\item подготовка отчёта по лабораторной работе.
\end{itemize}